\documentclass[journal]{IEEEtran}

\usepackage{graphicx}
% Insert additional usepackage commands here

\begin{document}
%
% paper title
% Titles are generally capitalized except for words such as a, an, and, as,
% at, but, by, for, in, nor, of, on, or, the, to and up, which are usually
% not capitalized unless they are the first or last word of the title.
% Linebreaks \\ can be used within to get better formatting as desired.
% Do not put math or special symbols in the title.
\title{Virtual Reailty Controllers and How They Effect Thoughts and Feelings}
%
%
% author name
\author{\IEEEauthorblockN{Dean Harland\\}
\IEEEauthorblockA{Falmouth Games Academy\\
Falmouth, UK\\
Student ID: 1507680\\
Email: DH185421@falmouth.ac.uk\\}
}
% The paper headers -- please do not change these, but uncomment one of them as appropriate
% Uncomment this one for COMP320
\markboth{COMP320: Research Review and Proposal}{COMP320: Research Review and Proposal}
% Uncomment this one for COMP360
% \markboth{COMP360: Dissertation}{COMP360: Dissertation}

% make the title area
\maketitle

% As a general rule, do not put math, special symbols or citations
% in the abstract or keywords.
\begin{abstract}
The study of the relationship between video games and aggression has been increasing in popularity as the more widespread and widely available games become. The gap in studies for virtual reality related video game aggression has inspired this dissertation. The study shown in this study
\end{abstract}

\section{Introduction}

\IEEEPARstart{V}{ideo} game aggression is a widely explored and debated field, in which debated that playing video games causes the player to be more prone to negative feelings and aggressive thoughts. Before the age of gaming, media had the spotlight on the matter of aggression but as this new technology developed there was more studies claiming that video games were a main source for aggression. Recent development in Virtual reality has raised questions, in does this level of immersion that comes from virtual reality have an increased effect on video game aggression. As this technology is in its infancy there is not a lot research in the matter of video game aggression linked to virtual reality. As the payer is more immersed in the game it would be hypothesised that the actions the player takes in the game would affect them in a deeper manner.
Understanding the correlation between video games and aggression is a big factor in helping inform studies of indirect causes on how and what video games cause. Using the information from papers on video game aggression will be a foundation to further understanding how controllers can also cause aggressive thoughts and feelings.




\section{Research Questions}

\section{Literature Review}

    \subsection{Video Game Aggression}
   
   \subsubsection{Craig Anderson}
         One influential contributor to the scientific literature in the area of video games on aggressive behavior is Craig Anderson. Craig Anderson has released to 2 meta analyses, the first one titled "Effects of Violent Video Games on Aggressive Behavior, Aggressive Cognition, Aggressive Affect, Physiological Arousal, And Prosocial Behavior: A Meta-Analytic Review of the Scientific Literature"\cite{Craig2001}.  In this report Craig Anderson reviewed 35 reports with a total of 4,262 participants with under half under the age of 18. Over the span of 33 tests of 3,033 participants results showed that there was a link with increased aggression. The research showed that the link was strong across experimental versus non experimental designs and also gender and age, which demonstrates that video games do result in increased real world aggression. Craig Anderson's second meta analyses improved significantly because of the increased number research available and improved methodologies. Titled Violent Video Games Effects on Aggression, Empathy, and Prosocial Behavior in Eastern and Western Countries: A Meta-Analytic Review \cite{Craig2010}, Craig Anderson reviewed 130 reports with a total of 130,000 participants. As with his first study he discovered that there was a link between video games and aggression short term and long term but also showing how it affected things such as negative effects on prosocial behavior, physiological arousal and the decline in empathy and increase in desensitization to violence. Debating whether or not Craig Anderson's second paper should be so revered brings out a small number of problems. Anderson excluded studies such as the physiological arousal of participants who played nonviolent video games as well as violent video games. Anderson also used a number of unpublished papers in his work, although the sheer amount of published papers outweigh it does give variety of unpublished papers.
    
    \subsubsection{Ferguson}
        Christopher Ferguson meta analysis titled The Public Health Risks of Media Violence: A Meta-Analytic Review \cite{Ferguson} had multiple criticisms of Andersons 2001 paper. In Ferguson’s paper he excluded all unpublished papers and any papers before 1998 as said they may pollute results. Ferguson investigated 27 studies of unspecified amount of participants with only 15 of them dealing with video games. The studies show that there is a minimal correlation between video games and aggressive behaviour and that dependent on the methodology affects the effects size and strength of the correlation. Although both of Anderson’s paper has a greater sample size compared to Ferguson this should not disprove Ferguson’s findings due to limitations.
    \subsubsection{Unbiased views}
        Another paper that challenges other theories of media impacts on violence to see if evidence was consistent with them is titled “Violent Video Games and Aggression: Why Can't We Find Effects?”\cite{Sherry}. He came to the conclusion that there was minimal effect on aggression from violent games and suggests that the types of methodologies amplifies previous studies. These types of studies play a big part in reinforcing other studies in a non biased method.  Similar to John Sherry’s paper, Greitemeyer published a paper titled “Video Games Do Affect Social Outcomes: A Meta-Analytic Review of the Effects of Violent and Prosocial Video Game Play”\cite{Greitemeyer} in which Greitemeyer discusses the positives and the negatives effects of videogames. Reviewing 98 studies of 36,965 participants with reliable outcomes from correlational, experimental and longitudinal studies. Studies with longitudinal designs had the lowest effects sizes while the experimental design has the highest effect size which is coherent with John Sherry’s study. Over the course of the studies Greitemeyer reveals that not only does that violent video games increase violent behaviour but prosocial video games increased prosocial behavior.
    \subsubsection{APA}
        In 2005 The American Psychological Association(APA) released a statement saying that all forms of violent media has a increased effect on aggressive behaviour\cite{APA}. In response to this Ferguson led an international group of 225 media scholar, criminologists and psychologists to suggest that they revise their statement because of recent literature and methodological flaws in the past\cite{APALetter}. 

There is a lot of evidence from multiple sources that state that video games do cause aggression, but with the flaws in the methodology it suggests that results are exaggerated. Where Craig Anderson papers state there is a significant increase in aggression, Ferguson does not deny there is no increase in aggression but the effects are small enough to cause no lasting effects. Although some individuals are more prone to aggression and it could be different for each individual studies suggests that there is still a correlation over all types of individuals. Rather than studying whether or not violent games cause aggression, perhaps studies should be more directly focused on what parts of the game causes this aggression.






    \subsubsection{Taylor Aggression Paradigm}
        The paper “This is your brain on violent video games”\cite{Christopher} goes in depth on how video games causes the player to become desensitized and because of this it leads to aggressive behaviour. Desensitization is where by the repeated exposure of violence results in a moderate acclimation to what once was negative thoughts and emotional responses to the violence \cite{Tamara}. In the study participants were told the study regarded the effects of video games on visual perception and reaction time. Concealing the true nature of their study helped to improve the validity of the results, where the player would unbenounced try and withhold feelings of the game. After the participant played either a nonviolent or violent game they were then shown pictures of either violent or nonviolent pictures, each picture only being shown for one second. In between these two stages they were told to listen to blast of white noise. Following these games and pictures they were then asked to play a competitive reaction time task also know as the Taylor Aggression Paradigm, stated by Giancol \& Zeichner,1995 to be a reliable and valid measure of aggression. They were told that they and an opponent would have to press a button as fast as they could and whoever was slower would have a blast of white noise. Before each trial the participant were asked to set the level of the white noise the opponent would receive, ranging from level 1 (60dB) to Level 10 (105dB) they were also given the choice of how long the opponent would hear this sounds for ranging from 0 to 2.5 seconds. The results of the findings did show that participants who had previously a lot of violent video game exposure and those who were shown the violent images and played the violent video games were more aggressive. Setting the opponent white noise at a higher setting than those who were exposed to non violent games. The experiment although with some weaknesses has informed that after the exposure of violent video games and images it does leave traces of anger, also the use of violent images afterwards could be used in it’s own study comparing the two to see if the violence is from the games or the images . The use of the Taylor aggression paradigm was a suitable way to measure the aggression from the participants, allowing a methodical way to measure and record the results. Although other outside factors could change the participants mood before taking this test. When participants are shown images outside of the game, it moves away from just video game aggression. The Taylor test used in \cite{YoungWomen} \cite{ChristopherR} shows that there is no standardized way of interpreting it, which could lead to careless use of the measure and capitalization on chance.

Finding the correct way to measure and collect the findings that will be collected in this dissertation will be a vital part, the use of the Taylor Aggression Paradigm has potential in other fields but instead will be used to search out more appropriate ways of testing aggression.


    \subsection{Virtual reality Aggression}
        With the vividness and interactivity of video games allegedly causing aggressive thoughts and feelings, by the simultaneous use of audible, visual and haptic signals. You can expect that virtual reality video games will far surpass this by adding a depth to each of these orienting systems\cite{Tamborini}. Virtual reality also increases the range of control the player has by using more natural actions to control with, and if the vividness and interactivity lead to a greater immersions\cite{Steuer} then both of these experiences should be significantly higher in virtual reality. Because the controller itself uses a more natural way and therefore the player does not need to learn a new set of controls, this means that it reduces the need for previous gaming experience and the figurative gap between the real world and the game. As explained by Noah Schaffer to be known as game usability\cite{Schaffer}. The Nintendo Wii controller is a well known console that uses this game usability well, with the Wii’s increase in popularity it has caused developers to create new ways to integrate and utilize the controller's unique capabilities. To enhance the realism of these controllers accessories were created such as the baseball bat and tennis racket. Relating back to Tamborini showing that pre-Wii research showed a positive relationship between the natural motion controllers and aggressive outcomes during violent games\cite{Tamborini}. Which is suggesting that these technological advancements could be making it easier for players to imitate the games content and enjoy the natural motion during violent video games. In one paper by Skalski, he tested different types of controller to play a car racing game and the steering wheel which was the most natural one\cite{Skalski}.





    \subsection{Research Methodologies}
    
        \textbf{Measurement of Aggression}
        
                Buss defined aggression as “a response that delivers noxious stimuli to another organism.”\cite{Buss}, he also put forward that aggression can be classified with the combination of three dichotomous variables. These are physical versus verbal, direct versus indirect, and active versus passive. There are 8 possible combinations for these variables, the four passive types are not common in experimental studies\cite{Bushman}, not measuring the passive types allows studies to focus on the active types for a more detailed findings. In methodology in the study of aggression they measure the 4 active types of human aggression. 

    \subsubsection{Direct Physical Aggression}
        The first is direct physical aggression, it is measured with the aggression machine paradigm\cite{Buss}. The participant and a associate who is working with the experiment are briefed that the experiment is to do with the effects of punishment on learning. With a fake lottery the real participant is set as the teacher and the associate is to be the learner. The teacher gives the stimulus materials to the associate who allegedly attempts to learn it, in some of these experiments the participant is purposely angered by the associate before the beginning of the learning. When the associate answers wrong on a trial, the teacher then punishes the associate with an electric shock, with different options available the teacher can choose the duration and intensity of the shock. Usually measured 1-10, 1 being a tingle to 10 being extreme pain, the duration and intensity is then recorded as the measures of aggression. Angering the participant in different ways could be an interesting to see what causes higher levels of aggression and open a new avenue of aggression, but with this type of experiment it seems that punishing someone for answering a wrong question doesn't have much reasoning and would depend on the teacher's personality.
        A similar method of studying physical aggression is berkowitz’s paradigm in which the participant and associate are asked to evaluate each other\cite{Berkowitz}. First of all the associate evaluates the participant's solution, half of the participants gain a good evaluation and the other half a negative one, evaluation is judge by 1 to 10 electric shocks(1 indicating a good evaluation and 10 being bad). The participant is then exposed to stimuli, either violent or nonviolent then they are then asked to evaluate the associate, this is then measured by how many shocks are given out. 

    \subsubsection{Indirect Physical Aggression}
        Barnett measures indirect physical aggression by recording how the participant takes the associates belongings\cite{Barnett}. In the study Barnett had participants in charge of the associates money, in which whenever the associate answered a question wrong the participant would deduct between 0 and 9 cents from the associate. Showcasing how aggression can affect how participants indirectly cause the associate harm.

    \subsubsection{Direct Verbal Aggression}
        Direct verbal aggression is usually measured by recording the experiment and counting the the times the participant negatively states something. An example of this is in the Contagion of aggression study where participants evaluate opinions of the associate on a multitude of topics\cite{Wheeler}. The majority of the time the associate would state objectionable statements to provoke the participant, the participant was then given the chance to evaluate and comment on the associate's views. The associate was in earshot so that the participant could make direct verbal attacks against the associate, it was then assessed whether or not the participant made aggressive remarks about the associate.

    \subsubsection{Indirect Verbal Aggression}
        Indirect verbal aggression is commonly done by the associate provoking the participant and then the participant secretly evaluating the associate with the knowledge that their evaluation will affect them in a certain way. Rohsenow and Bachorowskiexperiment had a participant draw a circle on a piece as slow as they could\cite{Rohsenow}. After they completed this task an experimenter bursts into the room, informs the participant he is a supervisor who has been observing the experiment and states that "Obviously, you don't follow instructions. You were supposed to trace the circle as slowly as possible without stopping but you clearly didn't do this. Now I don't know if we can use your data.” He then commands the participant to do it again. When the participant finishes, he has to complete an evaluation with a seven point scaled for every member of the laboratory staff including the impolite supervisor. The evaluations were then sealed and allegedly sent off to be used in future hiring decisions consequently negatively affecting the chances of the staff.

    \textbf{My Methods}
    
    An occurring theme in measuring aggression is having a variable in which the participant can express their aggression towards, Buss and Berkowitz both make use of inflicting physical pain on the associate through the use of electric shocks. Using methods of recording direct verbal aggression during the experiment along with a similar berkowitz’s paradigm after the experiment would allow me to measure the level of aggression from the participant.EXPLAIN MORE HERE HOW YOU ARE GOING TO MEASURE

\section{Preliminary Results}
\section{Conclusion}
The conclusion goes here.

% references section

\bibliographystyle{IEEEtran}
\bibliography{reference}

% Appendices

\appendices
\section{First appendix}
Appendices are optional. Delete or comment out this part if you do not need them.

% that's all folks
\end{document}